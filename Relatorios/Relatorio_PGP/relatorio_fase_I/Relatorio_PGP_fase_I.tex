\documentclass[a4paper]{report}

\usepackage[a4paper, total={6in, 9.5in}]{geometry}
\usepackage{etoolbox}% http://ctan.org/pkg/etoolbox
\makeatletter
\patchcmd{\@makechapterhead}{\vspace*{50\p@}}{}{}{}% Removes space above \chapter head
\makeatother

\usepackage[portuges]{babel}
\usepackage[latin1,utf8]{inputenc}
\usepackage{graphicx}
\usepackage{hyperref}
\usepackage{titlesec}
\usepackage{tabularx}

\titleformat{\chapter}{\normalfont\huge}{\thechapter.}{20pt}{\huge\bf}

\begin{document}
\title{\textbf{Planeamento e Gestão de Projetos}\linebreak \linebreak 2017-2018\linebreak \linebreak {\huge \textbf{Relatório FASE I}}}
\date{}
\author{
%	\begin{flushleft}
	\textbf{Grupo 001}\\
	\textbf{Autores:}\\
	André Nunes, fc43304\\
	Ana Catarina Sousa, fc48301\\ 
	Hugo Filipe Curado, fc48761\\ 
	Patrícia Jesus, fc46593\\
	Pedro Duarte Neto, fc48758
%	\end{flushleft}
}
\maketitle
\tableofcontents
\chapter{Introdução}
Este projeto tem como a objetivo criar uma plataforma de armazenamento, consulta e gestão de informação relativa aos acessos aos edifícios e salas da Faculdade de Ciências da Universidade de Lisboa, tendo em conta o hardware existente no espaço em causa e as necessidades do mesmo.

Esta plataforma será disponibilizada numa solução hibrida, integrando recurssos em cloud (AWS) e recursos locais.

A necessidade de um levantamento rigoroso das atuais soluções foi realizado na cadeira de Conceção do Produto no ano letivo 2016/2017. Com este levantamento das necessidades dos futuros utilizadores e das respetivas soluções já realizado ganhamos algum tempo sendo só necessário melhorar alguns aspetos.

Em relação ao projeto em causa, no minimo, terão que ser sustentadas três funcionalidades, uma para cada grupo de utilizadores:
\begin{enumerate}
\item Aos administradores da plataforma, consulta de quem esteve presente num edifício ou numa sala em determinados períodos temporais %fica a faltar secretaria e seguranças
\item Aos professores da FCUL, consulta dos alunos presentes nas suas aulas
\item Aos alunos da FCUL, consulta do seu histórico de acessos
\end{enumerate}

Passivamente o sistema sustentará, no minimo, os seguintes requisitos:
\begin{enumerate}
\item Tolerar a falha de um qualquer componente de hardware com uma redução mínima de desempenho e sem perda de dados;
\item Tolerar uma falha catastrófica com um período de indisponibilidade não superior a 24h, sendo admissível apenas a perda de dados que tenham sido introduzidos no sistema nas últimas 24h;
\item Ser escalável e modular, por forma a suportar facilmente a adição e remoção de hardware para fazer face a picos de utilização que se prevê que ocorram em determinados momentos.
\item confidencialidade de dados, tendo em conta a sensibilidade dos dados a manipular.
\end{enumerate}
\chapter{Âmbito do projeto}
\section{Requisitos funcionais}
\section{Requisitos não-funcionais}
\section{Informação de entrada e saída}
\chapter{Planeamento}
\section{Estimativas}
\subsection{Esforço disponível}
Elementos da equipa e respectivas disponibilidades:\\

\begin{tabularx}{\textwidth}{XX}
	André = 35\%    & Patricia = 40\% \\
	Catarina = 35\% & Pedro = 35\%    \\
	Hugo = 35\%     &
\end{tabularx}
\linebreak\linebreak
Tendo em conta a disponibilidade de cada elemento a equipa terá $0.35*4 + 0.4 =1.8$ pessoas\linebreak
Com uma duração prevista de 5 meses teremos um \textbf{esforço disponivel} de $1.8*5=9$ Pessoas/Mês
\subsection{Dados históricos}
\subsubsection*{André, Catarina, Hugo e Pedro} 
\begin{description}
\item[Projecto de Programação I] em que dados 2 ficheiros texto era produzido um ficehiro com a informação pretendida, para tal os dois ficheiros eram lidos e manipulados em memoria. O projecto teve uma \textbf{duração aproximada de um mês} e cada grupo desenvolveu, aproximadamente \textbf{100 linhas de codigo} em \textbf{python}.

\item[Projecto de Programação II] em que dado um ficheiro csv, eram produzidos gráficos com a informação mais relevanto do ficheiro em causa. O projecto teve uma \textbf{duração aproximada de um mês} e cada grupo desenvolveu, aproximadamente \textbf{100 linhas de código} em \textbf{python}, destaque para o uso do modulo pylab.

\item [Projeto de Introdução às Tecnologias Web] consistia em desenvolver um website a correr localmente só em browser. O tema era o jogo da forca. O projeto teve a \textbf{duração aproximada de dois meses e meio} e cada grupo desenvolveu aproximadamente \textbf{300 linhas de código} usando \textbf{HTML}, \textbf{CSS} e \textbf{Javascript}.

\item[3 mini-projetos de Sistemas Operativos]\mbox{}
	\begin{description}
		\item[O primeiro projeto] tratava de ficheiros duplicados em sistemas de ficheiros. Foi desenvolvido em \textbf{shell script}, teve \textbf{duração aproximada de três semanas} e \textbf{100 linhas de código}.
		
		\item [O segundo mini-projeto] centrava-se em cifrar/decifrar ficheiros usando programação paralela. Teve uma \textbf{duração de três semanas}, desenvolvido em \textbf{python} e teve \textbf{200 linhas de código}.
		
		\item[O terceiro mini-projeto] foi um aprimoramento do segundo em que se acrescentava a escrita em binário num ficheiro (log), se contabilizava o tempo de execução e se detetavam sinais (Ex.: SIGINT). Teve uma \textbf{duração de três semanas} e teve \textbf{aproximadamente 80 linhas de código}. Desenvolvido em \textbf{python}.
	\end{description}
	\item[Projecto de Sistemas Inteligentes] foi desenvolvido o jogo dos peões e respectivas funções de avaliação, teve \textbf{aproximadamente duração de 1 mês} e contou com \textbf{200 linhas de codigo} em \textbf{python}.
\end{description}

\subsubsection*{André, Catarina, Hugo, Patrícia e Pedro} 
\begin{description}
    \item[Projeto de Programação Centrada em Objectos] desenvolvido em duas fases, teve como objetivo o desenvolvimento uma plataforma de apoio a um restaurante, desenvolvido em \textbf{Java}. \textbf{200 linhas de código} e uma \textbf{duração de dois meses}.
    
    \item[Projeto de Bases de Dados] desenvolvido em duas fases, em que a primeira foi o desenho conceptual do enunciado e a segunda foi o mapeamento do desenho em SQL(DDL E DML). Teve uma \textbf{duração de 2 meses} e um \textbf{código de 250 linhas}.
    \item[4 mini-projetos de Aplicações Distribuidas] desenvolvidos ao longo do semestre.
    \begin{description}
    	\item[O primeiro projeto] foi o desenvolvimento de um servidor com recursos a serem disponibilizados, incluindo os respectivos locks, teve  \textbf{duração de 3 semanas} e \textbf{aproximadamente 200 linhas de codigo}.
    	\item[O segundo projeto] foi um incremento do primeiro quanto às funcionalidades, tendo em conta a forma de comunicação e apresentação de conteudos, teve a \textbf{duração de 2 semanas} e \textbf{aproximadamente 200 linhas de codigo}.
    	\item[O terceiro projeto] baseado num serviço WEB para um sistema simplificado, usou-se \textbf{sqlite3} e \textbf{flask}, teve duração de \textbf{3 semanas} e \textbf{aproximadamente 200 linhas de codigo}.
    	\item[O quarto projeto] foi um aprimoramento do terceiro e consistiu em fazer a comunicação segura do serviço WEB, HTTPS, chave publicas e privadas, teve \textbf{duração de 2 semanas} e \textbf{aproximandamente 100 linhas de codigo}.
    \end{description}
    \item[Projeto de Analise e Desenho de Software] foi dividido em 2 fases, teve por objetivo o desenvolvimento de um sistema para uma cadeia de supermercados. Para tal foi utilizado o unified process. Na 1ª fase foi feita a analise e desenho. Na 2ª fase foi realizada a  implementação e testes, teve \textbf{duração de 2 meses} e \textbf{200 linhas de codigo}, aproximadamente.
    
    \item[Projeto de Conceção do Produto] em que o tema é o mesmo que o da cadeira de Planeamento e Gestão de Projecto (PGP). , Neste projeto foi realizada a analise do problema a abordar em PGP e posteriormente no projeto final.
\end{description}

\subsubsection*{Catarina, Hugo e Pedro} 
\begin{description}
    \item[Projeto de Interação de Computadores] sobre uma aplicação tablet de encomenda de comida, ao longo do projeto foram desenvolvidos protótipos, questionários e por fim foi desenvolvido um protótipo funcional. Este projeto teve \textbf{duração aproximada de 2,5 meses} e um \textbf{código de 200 linhas}. Foi desenvolvido em \textbf{HTML}, \textbf{CSS} e \textbf{Javascript}.
\end{description}

\subsubsection*{Patricia} 
\begin{description}
    \item[Projeto de Interação de Computadores] sobre uma aplicação tablet de gestão de uma cozinha, ao longo do projeto foram desenvolvidos protótipos, questionários e por fim foi desenvolvido um protótipo funcional. Este projeto teve \textbf{duração aproximada de 2,5 meses} e um \textbf{código de 500 linhas}. Foi desenvolvido em \textbf{HTML}, \textbf{CSS} e \textbf{Javascript}.
\end{description}

\subsubsection*{Catarina, Hugo e Pedro}
\begin{description}
    \item[Projeto de Aplicações e Serviços Web] consistiu em desenvolver uma aplicação Full Stack para jogos de poker online, teve uma \textbf{duração de dois meses e meio}, cada projecto teve aproximadamente \textbf{500 linhas de codigo}. Para o desenvolvimeto foram usadas as linguagens \textbf{php}, \textbf{javascript}, \textbf{html} e \textbf{css} e como framework \textbf{CodeIgniter}.
\end{description}

\subsubsection*{André}
\begin{description}
    \item[Projeto de Aplicações e Serviços Web] consistiu em desenvolver uma aplicação Full Stack para jogos de poker online, teve uma \textbf{duração de dois meses e meio}, teve aproximadamente \textbf{500 linhas de codigo}. Para o desenvolvimeto foram usadas as linguagens \textbf{javascript}, \textbf{html} e \textbf{css} com recurso ao MEAN Stack.
\end{description}

\subsubsection*{Patricia}
\begin{description}
    \item[Projeto de Aplicações e Serviços Web] consistiu em desenvolver uma aplicação Full Stack para leilões, teve uma \textbf{duração de dois meses e meio}, cada projecto teve aproximadamente \textbf{6500 linhas de codigo}. Para o desenvolvimeto foram usadas as linguagens \textbf{php}, \textbf{javascript}, \textbf{sql}, \textbf{html} e \textbf{css} .
\end{description}

\subsection{Estimativa de linhas de código}
\subsection{Estimativa COCOMO}
\subsection{Análise crítica}
\section{Recursos}
\section{Processo de desenvolvimento de software}
\section{Organização da equipa}
\section{Planeamento do Projeto}
\section{Gestão de Riscos}
\chapter{Conclusão}
\chapter{Bibliografia }
\end{document}
