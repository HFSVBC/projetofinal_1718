\documentclass[a4paper]{report}

\usepackage[a4paper, total={6in, 9.5in}]{geometry}
\usepackage{etoolbox}% http://ctan.org/pkg/etoolbox
\makeatletter
\patchcmd{\@makechapterhead}{\vspace*{50\p@}}{}{}{}% Removes space above \chapter head
\makeatother

\usepackage[portuges]{babel}
\usepackage[latin1,utf8]{inputenc}
\usepackage{graphicx}
\usepackage{hyperref}
\usepackage{titlesec}
\usepackage{tabularx}

\titleformat{\chapter}{\normalfont\huge}{\thechapter.}{20pt}{\huge\bf}

\begin{document}
\title{\textbf{Planeamento e Gestão de Projetos}\linebreak \linebreak 2017-2018\linebreak \linebreak {\huge \textbf{Relatório FASE I}}}
\date{}
\author{
%	\begin{flushleft}
	\textbf{Grupo 001}\\
	\textbf{Autores:}\\
	André Nunes, fcxxxxx\\
	Ana Catarina Sousa, fc48301\\ 
	Hugo Filipe Curado, fc48761\\ 
	Patrícia Jesus, fcxxxxx\\
	Pedro Duarte Neto, fc48758
%	\end{flushleft}
}
\maketitle
\tableofcontents
\chapter{Introdução}
Este projeto tem como a conceção de uma plataforma de armazenamento, consulta e gestão de informação relativa aos acessos aos edifícios e salas da Faculdade de Ciências da Universidade de Lisboa, tendo em conta o hardware existente no espaço em causa e as necessidades do mesmo.
Esta plataforma será disponibilizada num ((ambiente cloud(AWS??) integrando recursos locais com recursos disponibilizados pela “cloud”)).

A necessidade de um levantamento rigoroso das atuais soluções foi feito na disciplina de Conceção do Produto no ano letivo 2016/2017. Com este levantamento das necessidades dos futuros utilizadores e das respetivas soluções podemos então aprimorar este aspecto em vez de o criar de base.

Em relação ao projeto em causa, no minimo, terão que ser sustentadas 3 funcionalidades, 1 para cada grupo de utilizadores.
\begin{enumerate}
\item Aos administradores da plataforma, consulta de quem esteve presente num edifício ou numa sala em determinados períodos temporais
\item Aos professores da FCUL, consulta dos alunos presentes nas suas aulas
\item Aos alunos da FCUL, consulta do seu histórico de acessos
\end{enumerate}

Passivamente o sistema sustentará, no minimo, os seguintes requisitos:
\begin{enumerate}
\item Tolerar a falha de um qualquer componente de hardware com uma redução mínima de desempenho e sem perda de dados;
\item Tolerar uma falha catastrófica com um período de indisponibilidade não superior a 24h, sendo admissível apenas a perda de dados que tenham sido introduzidos no sistema nas últimas 24h;
\item Ser escalável e modular, por forma a suportar facilmente a adição e remoção de hardware para fazer face a picos de utilização que se prevê que ocorram em determinados momentos.
\item confidencialidade de dados, tendo em conta a sensibilidade dos dados a manipular.
\end{enumerate}
\chapter{Âmbito do projeto}
\section{Requisitos funcionais}
\section{Requisitos não-funcionais}
\section{Informação de entrada e saída}
\chapter{Planeamento}
\section{Estimativas}
\subsection{Esforço disponível}
Elementos da equipa e respectivas disponibilidades:\\

\begin{tabularx}{\textwidth}{XX}
	André = 35\%    & Patricia = 40\% \\
	Catarina = 35\% & Pedro = 35\%    \\
	Hugo = 35\%     &
\end{tabularx}
\linebreak\linebreak
Tendo em conta a disponibilidade de cada elemento a equipa terá $0.35*4 + 0.4 =1.8$ pessoas\linebreak
Com uma duração prevista de 5 meses teremos um \textbf{esforço disponivel} de $1.8*5=9$ Pessoas/Mês
\subsection{Dados históricos}
\subsection{Estimativa de linhas de código}
\subsection{Estimativa COCOMO}
\subsection{Análise crítica}
\section{Recursos}
\section{Processo de desenvolvimento de software}
\section{Organização da equipa}
\section{Planeamento do Projeto}
\section{Gestão de Riscos}
\chapter{Conclusão}
\chapter{Bibliografia }
\end{document}