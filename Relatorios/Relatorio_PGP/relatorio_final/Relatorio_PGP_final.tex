\documentclass[a4paper]{report}

\usepackage[a4paper, total={6in, 9.5in}]{geometry}
\usepackage{etoolbox}% http://ctan.org/pkg/etoolbox
\makeatletter
\patchcmd{\@makechapterhead}{\vspace*{50\p@}}{}{}{}% Removes space above \chapter head
\makeatother

\usepackage[portuges]{babel}
\usepackage[latin1,utf8]{inputenc}
\usepackage{graphicx}
\usepackage{hyperref}
\usepackage{titlesec}
\usepackage{tabularx}
\usepackage{colortbl}
\usepackage{cite}
\usepackage[nottoc,numbib]{tocbibind}
\usepackage[normalem]{ulem}
 \useunder{\uline}{\ul}{}
 \usepackage{lscape}
\usepackage{longtable}
\usepackage{tabu}
\usepackage[table,xcdraw]{xcolor}
\usepackage[export]{adjustbox}[2011/08/13]


\titleformat{\chapter}{\normalfont\huge}{\thechapter.}{20pt}{\huge\bf}
\begin{document}
\title{\textbf{Planeamento e Gestão de Projetos}\linebreak \linebreak 2017-2018\linebreak \linebreak {\huge \textbf{Relatório Final}}}
\date{}
\author{
%	\begin{flushleft}
	\textbf{Grupo 001}\\
	\textbf{Autores:}\\
	André Nunes, fc43304\\
	Ana Catarina Sousa, fc48301\\ 
	Hugo Filipe Curado, fc48761\\ 
	Patrícia Jesus, fc46593\\
	Pedro Duarte Neto, fc48758
%	\end{flushleft}
}
\begin{figure}
	\begin{center}
		\includegraphics[scale=.2]{images/LogoFCUL.png}
	\end{center}
\end{figure}
\maketitle
\tableofcontents
\chapter{Introdução}
O objectivo para este trabalho é planear e conceber uma plataforma de controlo de acessos da Faculdade de Ciências. 
Atualmente o acesso aos diversos espaços da faculdade (salas de aula, laboratórios, gabinetes, garagem, etc.) não é adequado às necessidades da comunidade escolar da faculdade (alunos, professores, funcionários, etc.) portanto foi-nos pedida uma solução que pudesse minimizar este problema, tornando o acesso mais prático, de fácil gestão e manutenção, mantendo todas as suas funcionalidades e segurança.\\
No sistema actual por vezes não é possível fazer um controlo rigoroso sobre a entrada de pessoas não autorizadas em espaços reservados, o que leva a uma visão irrealista e pouco segura.\\
Outro problema encontrado foi o controlo de pessoas num determinado espaço, como por exemplo o número de alunos numa determinada sala ou auditorio.\\
Num caso mais extremo de uma catástrofe, não existe uma visão de quantas pessoas estão num espaço que é necessário evacuar.\\
De modo a minimizar o problema do controlo de acessos, foi-nos proposto desenvolver uma solução tecnologica.
Esta solução tem como base desenvolver um sistema acessível através de uma interface Web, disponivel num ambiente "cloud" e de facil integração com o hardware existente na faculdade.\\
É necessário também que este sistema seja seguro e disponivel. Devido à sensibilidade dos dados da aplicação, tais como nome dos alunos e professores, horários, o sistema deve usar mecanismos de segurança adequados de modo a manter a confidencialidade dos dados e a estar protegido contra vulnerabilidades e ataques\\
Em relação à disponibilidade, o sistema deverá tolerar a falha de de um qualquer componente de hardware com uma redução mínima de desempenho e sem perda de dados, e tolerar uma falha catastrófica com um período de indisponibilidade não superior a 24h, sendo admissível apenas a perda de dados que tenham sido introduzidos no sistema nas últimas 24h.\\
A análise de requisitos, fontes de informação e tratamento da informação foram efectuadas na disciplina de Conceção de Produto no semestre passado, e é com base nessa informação recolhida que desenvolvemos esta solução.\\
Posto isto é necessário primeiro planificar quais os requisitos do sistema, as necessidades tecnológicas, riscos e organização da equipa. É também necessário escolher um modelo de desenvolvimento para o projecto e planear a execução do mesmo.\\


\chapter{Âmbito do projeto}
\section{Requisitos funcionais}
\begin{enumerate}
\item Aluno
	\begin{enumerate}
	\item Consulta de histórico de acessos pessoal
    \item Horário do curso frequentado
    \item Horário dos espaços livres e a que grupo pertence (Ex.: grupo = Departamento de Informática)
	\end{enumerate}
\item Funcionário de Departamento
	\begin{enumerate}
	\item Consulta de histórico de acessos pessoal
    \item Consulta de salas disponíveis
    \item Marcação de salas
	\end{enumerate}
\item Secretaria
	\begin{enumerate}
    \item Consultar e alterar presenças nas aulas, mediante justificação válida
	\end{enumerate}
\item Professor
	\begin{enumerate}
	\item Consulta dos alunos presentes nas suas aulas
    \item Horário de aulas e "compromissos"
	\item Consulta de histórico de acessos pessoal
    \item Consulta de salas disponíveis
    \item Marcação de salas para avaliações contínua e aulas extra
	\end{enumerate}
\item Seguranças
	\begin{enumerate}
    \item Consulta de histórico de acessos
    \item Consulta de salas disponíveis
    \item Criar acesso para visitantes
    \item Criar eventos de incidentes / desastres / perdidos e achados
    \item Criar acessos temporários no caso de perda de cartão
	\item Consulta de quantas pessoas estiveram presentes num edifício ou numa sala num determinado espaço de tempo
    \item Consulta de quem esteve presente num edifício ou numa sala num determinado espaço de tempo, em caso de desastre, validado por mais que uma pessoa com um nível de Administrador de Sistema / Administrador-Chefe
    \item Bloqueio de acessos em caso de necessidade, Lock Down
    \item Desbloqueio do estado Lock Down, validado por mais que uma pessoa com um nível de Administrador de Sistema
    \item Consultar acessos ao parque de estacionamento
	\end{enumerate}
\item Administrador
	\begin{enumerate}
    \item Consulta de quantas pessoas estiveram presentes num edifício ou numa sala num determinado espaço de tempo
    \item Registo das salas disponíveis num determinado espaço temporal
    \item Registo de novos utilizadores
	\item Consulta de histórico de acessos pessoal
    \item Alteração de estatutos, inferior ao seu nível de acesso
    \item Associar cartão de aluno no sistema informático da dsi
    \item Criar acesso para visitantes
    \item Alterar acesso a salas e laboratórios
    \item Consulta de salas disponíveis
    \item Consultar acessos ao parque de estacionamento
    \end{enumerate}
\item Administrador-Chefe (Ex. Zé Fernandes)
	\begin{enumerate}
	\item Consulta de quem esteve presente num edifício ou numa sala num determinado espaço de tempo, em caso de um incidente
    \item Consulta de quem esteve presente num edifício ou numa sala num determinado espaço de tempo, em caso de desastre, validado por mais que uma pessoa com um nível de Administrador de Sistema
    \item Todas as funcionalidades do Administrador
	\end{enumerate}
\item Administrador de Sistemas
	\begin{enumerate}
    \item Alteração de todos os estatutos
    \item Todas as funcionalidades do Administrador
    \end{enumerate}
\end{enumerate}
\section{Requisitos não-funcionais}
\begin{enumerate}
\item Acessível através de interface web na cloud, compatível com qualquer browser e dispositivo.
\item Facilidade de integração com o hardware existente.
\item Necessidade de ligação constante à internet.
\item Confidencialidade dos dados.
\item Um tipo de utilizador terá acesso a uma visão global da base de dados em caso de necessidade (Ex.: Desastre natural ou roubo numa determinda sala)
\item Tolerar a falha de um qualquer componente de hardware com uma redução mínima de desempenho e sem perda de dados
\item Tolerar uma falha catastrófica com um período de indisponibilidade não superior a 24h, sendo admissível apenas a perda de dados que tenham sido introduzidos no sistema nas últimas 24h
\item Ser escalável e modular, por forma a suportar facilmente a adição e remoção de hardware para fazer face a picos de utilização que se prevê que ocorram em determinados momentos.
\item Integração no sistema de autenticação atualmente presente na faculdade.
\end{enumerate}
\section{Informação de entrada e saída}
\subsubsection*{Entrada inicial de informação}
    Os dados sobre os utilizadores são gerados automaticamente por um serviço de simulação da interação de um grupo de pessoas com um edifício, sendo depois acedidos por JSON.
	Os dados simulados são dados referentes à implementação de um novo sistema de controlo de acessos. Este sistema é composto por aquesição do id do sítio por onde o utilizador passou, um timestamp que representa a hora a que passou e o id do utilizador. Estes pontos de controlo estaram inseridos antes e depois da porta ou entrada de qualquer espaço da faculdade. Sendo depois processados pela nossa aplicação. Para os dados relativos aos utilizadores, tais como nome, número da faculdade, curso, entre outros, é necessário haver uma ligação à base de dados académica da faculdade de modo a termos acesso a essa informação. Para a informação sobre horários, salas e espaços é necessário uma ligação ao sistema Fenix da Faculdade de Ciências. Para a informação sobre o estacionamento na faculdade é necessário uma ligação ao sistema Verex da faculdade.
\subsubsection*{Saídas de informação}
	Os dados de saída seram os dados relativos ao histórico de acesso de utilizadores, informação sobre a disponibilidade e ocupação dos espaços, incidentes e perdidos e achados, permissões de cada utilizador.  
\chapter{Planeamento}
\section{Estimativas}
\subsection{Esforço disponível}
Início do Projeto a 5 de Fevereiro com conclusão prevista a 7 de Julho, o que resulta num total de 5 meses.\\
Elementos da equipa e respetivas disponibilidades:\\

\begin{tabularx}{\textwidth}{XX}
	André = 35\%    & Patrícia = 40\% \\
	Catarina = 35\% & Pedro = 35\%    \\
	Hugo = 35\%     &
\end{tabularx}
\linebreak\linebreak
Tendo em conta a disponibilidade de cada elemento a equipa terá $0.35*4 + 0.4 =1.8$ pessoas\linebreak
Com uma duração prevista de \textbf{5 meses} teremos um \textbf{esforço disponível} de $1.8*5=9$ Pessoa Mes
\subsection{Dados históricos}
\subsubsection*{André, Catarina, Hugo e Pedro} 
\begin{description}
\item[Projeto de Programação I] em que dados 2 ficheiros texto era produzido um ficheiro com a informação pretendida, para tal os dois ficheiros eram lidos e manipulados em memoria. O projeto teve uma \textbf{duração aproximada de um mês} e cada grupo desenvolveu, aproximadamente \textbf{100 linhas de código} em \textbf{python}.\\\textbf{Esforço} = 2*0.2(1 cadeira em 5)pessoas* 1 mês=0.4 PM\\ \textbf{Produtividade} =100 LOC/ 0.4 PM =250 LOC/PM

\item[Projeto de Programação II] em que dado um ficheiro csv, eram produzidos gráficos com a informação mais relevante do ficheiro em causa. O projeto teve uma \textbf{duração aproximada de um mês} e cada grupo desenvolveu, aproximadamente \textbf{100 linhas de código} em \textbf{python}, destaque para o uso do modulo pylab.\\\textbf{Esforço} = 2*0.2(1 cadeira em 5)pessoas* 1 mês=0.4 PM\\ \textbf{Produtividade} =100 LOC/ 0.4 PM =250 LOC/PM

\item [Projeto de Introdução às Tecnologias Web] consistia em desenvolver um website a correr localmente só em browser. O tema era o jogo da forca. O projeto teve a \textbf{duração aproximada de dois meses e meio} e cada grupo desenvolveu aproximadamente \textbf{300 linhas de código} usando \textbf{HTML}, \textbf{CSS} e \textbf{Javascript}.\\\textbf{Esforço} = 3*0.2 pessoas*2.5 meses=1.5 P M\\ \textbf{Produtividade} = 300 LOC/1.5 PM= 200 LOC/PM

\item[3 mini-projetos de Sistemas Operativos]\mbox{}
	\begin{description}
		\item[O primeiro projeto] tratava de ficheiros duplicados em sistemas de ficheiros. Foi desenvolvido em \textbf{shell script}, teve \textbf{duração aproximada de três semanas} e \textbf{100 linhas de código}.		
		\item [O segundo mini-projeto] centrava-se em cifrar/decifrar ficheiros usando programação paralela. Teve uma \textbf{duração de três semanas}, desenvolvido em \textbf{python} e teve \textbf{200 linhas de código}.
		
		\item[O terceiro mini-projeto] foi um aprimoramento do segundo em que se acrescentava a escrita em binário num ficheiro (log), se contabilizava o tempo de execução e se detetavam sinais (Ex.: SIGINT). Teve uma \textbf{duração de três semanas} e teve \textbf{aproximadamente 80 linhas de código}. Desenvolvido em \textbf{python}.
	\end{description}\mbox{}\textbf{Esforço} = 3*0.2 pessoas*2meses=1,2PM\\ \textbf{Produtividade} = 380 LOC/1.2 PM = 317 LOC/PM
	\item[Projeto de Sistemas Inteligentes] foi desenvolvido o jogo dos peões e respetivas funções de avaliação, teve \textbf{aproximadamente duração de 1 mês} e contou com \textbf{200 linhas de código} em \textbf{python}.\\\textbf{Esforço} = 3*0.2pessoas*1mes =0.6 PM\\ \textbf{Produtividade} = 200 LOC/0.6 PM = 333 LOC/PM
\end{description}

\subsubsection*{André, Catarina, Hugo, Patrícia e Pedro} 
\begin{description}
    \item[Projeto de Programação Centrada em Objetos] desenvolvido em duas fases, teve como objetivo o desenvolvimento uma plataforma de apoio a um restaurante, desenvolvido em \textbf{Java}. \textbf{200 linhas de código} e uma \textbf{duração de dois meses}.\\\textbf{Esforço} = 2*0.2*2meses = 0.8PM\\ \textbf{Produtividade} = 200 LOC/0.8 PM = 250 LOC/PM
    
    \item[Projeto de Bases de Dados] desenvolvido em duas fases, em que a primeira foi o desenho conceptual do enunciado e a segunda foi o mapeamento do desenho em SQL(DDL E DML). Teve uma \textbf{duração de 2 meses} e um \textbf{código de 250 linhas}.\\\textbf{Esforço} = 3*0.2*2meses =1.2PM\\ \textbf{Produtividade} = 250 LOC/1.2 PM = 208 LOC/PM
    \item[4 mini-projetos de Aplicações Distribuídas] desenvolvidos ao longo do semestre.
    \begin{description}
    	\item[O primeiro projeto] foi o desenvolvimento de um servidor com recursos a serem disponibilizados, incluindo os respetivos locks, teve  \textbf{duração de 3 semanas} e \textbf{aproximadamente 200 linhas de código}.
    	\item[O segundo projeto] foi um incremento do primeiro quanto às funcionalidades, tendo em conta a forma de comunicação e apresentação de conteúdos, teve a \textbf{duração de 2 semanas} e \textbf{aproximadamente 200 linhas de código}.
    	\item[O terceiro projeto] baseado num serviço WEB para um sistema simplificado, usou-se \textbf{sqlite3} e \textbf{flask}, teve duração de \textbf{3 semanas} e \textbf{aproximadamente 200 linhas de código}.
    	\item[O quarto projeto] foi um aprimoramento do terceiro e consistiu em fazer a comunicação segura do serviço WEB, HTTPS, chave publicas e privadas, teve \textbf{duração de 2 semanas} e \textbf{aproximadamente 100 linhas de código}.
    \end{description}\mbox{}\textbf{Esforço} = 3*0.2*2,5meses=1.5PM\\ \textbf{Produtividade} = 700 LOC/1.5 PM = 467 LOC/PM
    \item[Projeto de Analise e Desenho de Software] foi dividido em 2 fases, teve por objetivo o desenvolvimento de um sistema para uma cadeia de supermercados. Para tal foi utilizado o unified process. Na 1ª fase foi feita a analise e desenho. Na 2ª fase foi realizada a  implementação e testes, teve \textbf{duração de 2 meses} e \textbf{200 linhas de código}, aproximadamente.\\\textbf{Esforço} = 3*0.2*2meses=1.2PM\\ \textbf{Produtividade} = 200 LOC/1.2 PM = 167 LOC/PM
    
    \item[Projeto de Conceção do Produto] em que o tema é o mesmo que o da cadeira de Planeamento e Gestão de Projeto (PGP). , Neste projeto foi realizada a analise do problema a abordar em PGP e posteriormente no projeto final.
\end{description}

\subsubsection*{Catarina, Hugo e Pedro} 
\begin{description}
    \item[Projeto de Interação de Computadores] sobre uma aplicação tablet de encomenda de comida, ao longo do projeto foram desenvolvidos protótipos, questionários e por fim foi desenvolvido um protótipo funcional. Este projeto teve \textbf{duração aproximada de 2,5 meses} e um \textbf{código de 200 linhas}. Foi desenvolvido em \textbf{HTML}, \textbf{CSS} e \textbf{Javascript}.\\\textbf{Esforço} = 3*0.2*2.5meses=1.5PM\\ \textbf{Produtividade} = 200 LOC/1.5 PM = 133 LOC/PM
\end{description}

\subsubsection*{Patrícia} 
\begin{description}
    \item[Projeto de Interação de Computadores] sobre uma aplicação tablet de gestão de uma cozinha, ao longo do projeto foram desenvolvidos protótipos, questionários e por fim foi desenvolvido um protótipo funcional. Este projeto teve \textbf{duração aproximada de 2,5 meses} e um \textbf{código de 500 linhas}. Foi desenvolvido em \textbf{HTML}, \textbf{CSS} e \textbf{Javascript}.\\\textbf{Esforço} = 3*0.2*2.5meses=1.5PM\\ \textbf{Produtividade} = 500 LOC/1.5 PM = 333 LOC/PM
\end{description}

\subsubsection*{Catarina, Hugo e Pedro}
\begin{description}
    \item[Projeto de Aplicações e Serviços Web] consistiu em desenvolver uma aplicação Full Stack para jogos de poker online, teve uma \textbf{duração de dois meses e meio}, cada projeto teve aproximadamente \textbf{500 linhas de código}. Para o desenvolvimento foram usadas as linguagens \textbf{php}, \textbf{javascript}, \textbf{html} e \textbf{css} e como framework \textbf{CodeIgniter}.\\\textbf{Esforço} = 3*0.2*2.5meses=1.5PM\\ \textbf{Produtividade} = 500 LOC/1.5 PM = 333 LOC/PM
\end{description}

\subsubsection*{André}
\begin{description}
    \item[Projeto de Aplicações e Serviços Web] consistiu em desenvolver uma aplicação Full Stack para jogos de poker online, teve uma \textbf{duração de dois meses e meio}, teve aproximadamente \textbf{1500 linhas de código}. Para o desenvolvimento foram usadas as linguagens \textbf{javascript}, \textbf{html} e \textbf{css} com recurso ao MEAN Stack.\\\textbf{Esforço} = 3*0.2*2.5meses=1.5PM\\ \textbf{Produtividade} = 1500 LOC/1.5 PM = 1000 LOC/PM
\end{description}

\subsubsection*{Patrícia}
\begin{description}
    \item[Projeto de Aplicações e Serviços Web] consistiu em desenvolver uma aplicação Full Stack para leilões, teve uma \textbf{duração de dois meses e meio}, cada projeto teve aproximadamente \textbf{6500 linhas de código}. Para o desenvolvimento foram usadas as linguagens \textbf{php}, \textbf{javascript}, \textbf{sql}, \textbf{html} e \textbf{css} .\\\textbf{Esforço} = 3*0.2*2.5meses=1.5PM\\ \textbf{Produtividade} = 6500 LOC/1.5 PM = 4333 LOC/PM
\end{description}
\pagebreak
\subsection{Estimativa de linhas de código}
Com base nos dados históricos de cada membro da equipa, foi feita a estimação das linhas de
código nas diferentes perspectivas (otimista, provável e pessimista). De modo a ter uma melhor noção mais realista, dividimos o projecto nas várias componentes e estimamos a sua dimensão. As linhas de código para o simuldor de dados não foram tidas em conta no calculo final da ELC, isto porque os dados seriam obtidos usando hardware disponível na faculdade caso seja para produção.

\begin{table}[h]
\centering
\label{my-label}
\begin{tabular}{l|l|l|l|l}
 & Otimista & Provável & Pessimista & Final \\ \hline
Servidor MEAN & 400 & 600 & 1000 & 633 \\ \hline
Controlador & 800 & 1000 & 1500 & 1050 \\ \hline
Views & 850 & 1200 & 1600 & 1208 \\ \hline
Servidor LAMP & 450 & 800 & 1200 & 808 \\ \hline
Controlador & 750 & 1000 & 1300 & 1008 \\ \hline
Modelos & 250 & 500 & 800 & 508 \\ \hline
Base de dados & 150 & 350 & 500 & 342 \\ \hline
Segurança & 150 & 500 & 800 & 492 \\ \hline
Simulador de dados & 350 & 500 & 800 & 525 \\ \hline
Total & 3800 & 5950 & 8700 & \textbf{6048}
\end{tabular}
\caption{Estimativa de Linhas de Código}
\end{table}

EAL Final = $\frac{Otimista + 4 * Provavel + Pessimista}{6}$ = 6048
\pagebreak
\subsection{Estimativa COCOMO}
Para este projeto optámos pelo modelo de COCOMO orgânico visto que:
\begin{itemize}
\item a compreenção do produto é alargada com base no conhecimento já obtido no projeto da cadeira de CP, 
\item a experiência em projetos semelhantes foi obtida ao longo da carreira académica de cada um dos elementos do grupo de trabalho, 
\item a nessecidade de conformidade com pré requisistos é básica, visto que os requesitos funcionais do projeto não são muito extensos, focando-se em duas funcionalidades principais, registo de presenças e controlo de acessos,
\item a necessidade de conformidade com interfaces externas é básica, visto que não nos foi dado acesso aos recursos já existentes, sendo estes dados simulados por um único recurso,
\item a necessidade de desenvolvimento comcorrente de hardware e software não existe dado que não é o foco do projeto o desenvolvimento de hardware compativel com o sistema a desenvolver e o projeto não irá ser integrado com o sistema já existente na faculdade,
\item a necessidade de estruturas de dados ou algoritmos inovadores pode ser considerada mínima visto que, apesar de ser nessário uma base de dados extensa para o backend não irá existir a necessidade algoritmos inovadores,
\item o interesse em terminar o produto cedo é média visto que o projeto tem um prazo de entrega fixo com duração máxima de 5 meses,
\item a dimensão do produto enquadra-se dentro do modo orgâninco sendo previstas 2,75 KLOC
\end{itemize}

\begin{table}[h]
\centering
\begin{adjustbox}{width=1\textwidth}
\begin{tabular}{|l|l|l|l|l|l|}
\hline
 & \multicolumn{5}{c|}{\textbf{Classificação}} \\ \hline
 & \textbf{Muito reduzido} & \textbf{Reduzido} & \textbf{Nominal} & \textbf{Elevado} & \textbf{Muito elevado} \\ \hline
\textbf{Atributos do produto} &  &  &  &  &  \\ \hline
\textbf{Fiabilidade do software necessária} & .75 & .88 & \cellcolor[HTML]{FFFE65}1.0 & 1.15 & 1.4 \\ \hline
\textbf{Dimensão da base de dados} &  & \cellcolor[HTML]{FFFE65}.94 & 1.0 & 1.08 & 1.16 \\ \hline
\textbf{Complexidade do produto} & .70 & .85 & 1.0 & \cellcolor[HTML]{FFFE65}1.15 & 1.3 \\ \hline
 &  &  &  &  &  \\ \hline
\textbf{Atributos do computador} &  &  &  &  &  \\ \hline
\textbf{Restrições ao tempo de execução} &  &  & \cellcolor[HTML]{FFFE65}1.0 & 1.11 & 1.3 \\ \hline
\textbf{Restrições ao armazenamento de dados} &  &  & \cellcolor[HTML]{FFFE65}1.0 & 1.06 & 1.21 \\ \hline
\textbf{Volatilidade da máquina virtual} &  & \cellcolor[HTML]{FFFE65}.87 & 1.0 & 1.15 & 1.3 \\ \hline
\textbf{Tempo disponível do computador} &  & .87 & \cellcolor[HTML]{FFFE65}1.0 & 1.07 & 1.15 \\ \hline
 &  &  &  &  &  \\ \hline
\textbf{Atributos dos indivíduos} &  &  &  &  &  \\ \hline
\textbf{Capacidade dos analistas} & 1.46 & 1.19 & 1.0 & \cellcolor[HTML]{FFFE65}.86 & .71 \\ \hline
\textbf{Experiência no desenv. De aplicações} & 1.29 & 1.13 & \cellcolor[HTML]{FFFE65}1.0 & .91 & .82 \\ \hline
\textbf{Capacidade de programação} & 1.42 & 1.17 & \cellcolor[HTML]{FFFE65}1.0 & .86 & .70 \\ \hline
\textbf{Experiência com a máquina virtual} & 1.21 & 1.1 & \cellcolor[HTML]{FFFE65}1.0 & .90 &  \\ \hline
\textbf{Experiência com a liguagem de prog.} & 1.14 & 1.07 & \cellcolor[HTML]{FFFE65}1.0 & .95 &  \\ \hline
 &  &  &  &  &  \\ \hline
\textbf{Atributos do projecto} &  &  &  &  &  \\ \hline
\textbf{Uso de práticas modernas de programação} & 1.24 & 1.1 & \cellcolor[HTML]{FFFE65}1.0 & .91 & .82 \\ \hline
\textbf{Uso de ferramentas de software} & 1.24 & 1.1 & \cellcolor[HTML]{FFFE65}1.0 & .91 & .83 \\ \hline
\textbf{Prazos de desenvolvimento} & 1.23 & \cellcolor[HTML]{FFFE65}1.08 & 1.0 & 1.04 & 1.1 \\ \hline
\end{tabular}
\end{adjustbox}
\caption{Tabela de valores de multiplicadores de esforço}
\label{my-label}
\end{table}

Com base no COCOMO intermédio usando o modelo orgânico e tendo em conta os multiplicadores de esforço retirados da tabela dos Slides da TP06 \cite{slideTP6} obtivemos os seguintes valores:\\\\ 
KLOC = $6048 / 1000 = 6.05$ \\
Esforço [E] = $a * KLOC^{b} * EAF <=> 3.2 * 6.05^{1.05} * 0.87 = 18.43 PM$\\
Duração [D] = $c * E^{d} * EAF <=> 2.5 * 10.80^{0.38} = 7.56 Meses$\\
Numero de Pessoas [N] = $E / D <=> N = 18.43 / 7.56 = 2.43 Pessoas$

\pagebreak
\subsection{Análise crítica}
\label{AC}

Depois de uma análise aos dados históricos e a disponibilidade de cada elemento do grupo concluimos que o projecto terá uma duração prevista de \textbf{5 meses} e um esforço de \textbf{9 PM}. Com os valores finais estimados das linhas de código obtivemos um KLOC de \textbf{6048}, sem contar com a implementação do simulador de dados. De acordo com os cáculos do COCOMO intermédio obtivemos um esforço de \textbf{18.43 PM} e uma duração de \textbf{7.56 meses}.\\
Perante estes valores podemos observar que a duração é muito superior ao tempo disponível que temos para a elaboração do projecto. Uma das possíveis causas para esta descrepância pode ser devido ao facto de que a estimação das linhas de código ter-se baseado em projectos de pequena dimensão e em tecnologias e arquiteturas diferentes. Visto que o esforço necessário para o projecto é de \textbf{18.43 PM}, mais de metade do esforço disponível do grupo e a duração ter ultrapassado mais de 20\% da duração disponível, decidimos aplicar o modelo intermédio do COCOMO a uma nova tabela de Estimativa de Linahs de Código. Esta nova tabela contempla todas as funcionalidades fundamentais pedidas no enunciado, garantindo assim que no final temos uma aplicação para apresentar.

\begin{table}[h]
\centering
\label{my-label}
\begin{tabular}{l|l|l|l|l}
 & Otimista & Provável & Pessimista & Final \\ \hline
Servidor MEAN & 150 & 250 & 500 & 275 \\ \hline
Controlador & 300 & 450 & 700 & 467 \\ \hline
Views & 300 & 500 & 800 & 517 \\ \hline
Servidor LAMP & 150 & 350 & 650 & 367 \\ \hline
Controlador & 350 & 450 & 800 & 492 \\ \hline
Modelos & 200 & 400 & 700 & 417 \\ \hline
Base de dados & 100 & 150 & 300 & 167 \\ \hline
Segurança & 100 & 200 & 500 & 233 \\ \hline
Simulador de dados & 350 & 500 & 800 & 525 \\ \hline
Total & 1650 & 2750 & 4950 & \textbf{2935}
\end{tabular}
\caption{Estimativa de Linhas de Código}
\end{table}
EAL Final = $\frac{Otimista + 4 * Provavel + Pessimista}{6}$ = 2935
KLOC = $2935 / 1000 = 2.935$ \\
Esforço [E] = $a * KLOC^{b} * EAF <=> 3.2 * 2.935^{1.05} * 0.87 = 8.62 PM$\\
Duração [D] = $c * E^{d} * EAF <=> 2.5 * 10.80^{0.38} = 5.66 Meses$\\
Numero de Pessoas [N] = $E / D <=> N = 8.62 / 5.66 = 1.52 Pessoas$\\\\

Com base nestes resultados obtivemos então um esforço de \textbf{8.62 PM} e uma duração de \textbf{5.66 meses}. Estes valores já são aceitáveis para a elaboração do projecto pois correspondem às previsões efectuadas da equipa. Na tabela acima estão descritas as linhas de código para realizar as funcionalidades necessárias para obter aprovação às disciplinas de PTI e PTR, incluíndo o mínimo de segurança e fiabilidade do sistema.

\pagebreak
\section{Recursos}
\begin{table}[h]
\centering
\begin{tabularx}{\textwidth}{X|c|c|c}
\textbf{Descrição} & \textbf{Disp. Média} & \textbf{Quando Necessário} & \textbf{Tempo Necessário} \\ \hline
André Nunes & 35\% & 80\%  & 8:00 horas \\ \hline
Ana Catarina Sousa & 35\% & 80\% & 8:00 horas \\ \hline
Hugo Filipe Curado & 35\% & 80\% & 8:00 horas \\ \hline
Patrícia Jesus & 40\% & 90\% & 8:00 horas \\ \hline
Pedro Duarte Neto & 35\% & 80\% & 8:00 horas \\
\end{tabularx}
\caption{Tabela de Recursos Humanos}
\label{TabRH}
\end{table}

\begin{table}[h]
\centering
\begin{tabular}{llllllll}
\multicolumn{1}{l|}{\textbf{Pessoa / Semana}} & \multicolumn{1}{l|}{\textbf{05-Feb}} & \multicolumn{1}{l|}{\textbf{12-Feb}} & \multicolumn{1}{l|}{\textbf{19-Feb}} & \multicolumn{1}{l|}{\textbf{26-Feb}} & \multicolumn{1}{l|}{\textbf{05-Mar}} & \multicolumn{1}{l|}{\textbf{12-Mar}} & \textbf{19-Mar} \\ \hline
\multicolumn{1}{l|}{\textbf{André Nunes}} & \multicolumn{1}{l|}{90,00\%} & \multicolumn{1}{l|}{90,00\%} & \multicolumn{1}{l|}{50,00\%} & \multicolumn{1}{l|}{50,00\%} & \multicolumn{1}{l|}{40,00\%} & \multicolumn{1}{l|}{20,00\%} & 25,00\% \\ \hline
\multicolumn{1}{l|}{\textbf{Ana Catarina Sousa}} & \multicolumn{1}{l|}{70,00\%} & \multicolumn{1}{l|}{70,00\%} & \multicolumn{1}{l|}{50,00\%} & \multicolumn{1}{l|}{50,00\%} & \multicolumn{1}{l|}{40,00\%} & \multicolumn{1}{l|}{20,00\%} & 20,00\% \\ \hline
\multicolumn{1}{l|}{\textbf{Hugo Filipe Curado}} & \multicolumn{1}{l|}{90,00\%} & \multicolumn{1}{l|}{90,00\%} & \multicolumn{1}{l|}{50,00\%} & \multicolumn{1}{l|}{50,00\%} & \multicolumn{1}{l|}{40,00\%} & \multicolumn{1}{l|}{20,00\%} & 20,00\% \\ \hline
\multicolumn{1}{l|}{\textbf{Patricia Jesus}} & \multicolumn{1}{l|}{90,00\%} & \multicolumn{1}{l|}{90,00\%} & \multicolumn{1}{l|}{40,00\%} & \multicolumn{1}{l|}{30,00\%} & \multicolumn{1}{l|}{40,00\%} & \multicolumn{1}{l|}{55,00\%} & 40,00\% \\ \hline
\multicolumn{1}{l|}{\textbf{Pedro Neto}} & \multicolumn{1}{l|}{70,00\%} & \multicolumn{1}{l|}{70,00\%} & \multicolumn{1}{l|}{50,00\%} & \multicolumn{1}{l|}{50,00\%} & \multicolumn{1}{l|}{35,00\%} & \multicolumn{1}{l|}{20,00\%} & 20,00\% \\
\textbf{} &  &  &  &  &  &  &  \\
\multicolumn{1}{l|}{\textbf{Pessoa / Semana}} & \multicolumn{1}{l|}{\textbf{26-Mar}} & \multicolumn{1}{l|}{\textbf{02-Apr}} & \multicolumn{1}{l|}{\textbf{09-Apr}} & \multicolumn{1}{l|}{\textbf{16-Apr}} & \multicolumn{1}{l|}{\textbf{23-Apr}} & \multicolumn{1}{l|}{\textbf{30-Apr}} & \textbf{07-May} \\ \hline
\multicolumn{1}{l|}{\textbf{André Nunes}} & \multicolumn{1}{l|}{25,00\%} & \multicolumn{1}{l|}{20,00\%} & \multicolumn{1}{l|}{30,00\%} & \multicolumn{1}{l|}{30,00\%} & \multicolumn{1}{l|}{20,00\%} & \multicolumn{1}{l|}{30,00\%} & 30,00\% \\ \hline
\multicolumn{1}{l|}{\textbf{Ana Catarina Sousa}} & \multicolumn{1}{l|}{25,00\%} & \multicolumn{1}{l|}{20,00\%} & \multicolumn{1}{l|}{40,00\%} & \multicolumn{1}{l|}{40,00\%} & \multicolumn{1}{l|}{20,00\%} & \multicolumn{1}{l|}{20,00\%} & 50,00\% \\ \hline
\multicolumn{1}{l|}{\textbf{Hugo Filipe Curado}} & \multicolumn{1}{l|}{25,00\%} & \multicolumn{1}{l|}{20,00\%} & \multicolumn{1}{l|}{30,00\%} & \multicolumn{1}{l|}{30,00\%} & \multicolumn{1}{l|}{20,00\%} & \multicolumn{1}{l|}{30,00\%} & 30,00\% \\ \hline
\multicolumn{1}{l|}{\textbf{Patricia Jesus}} & \multicolumn{1}{l|}{25,00\%} & \multicolumn{1}{l|}{50,00\%} & \multicolumn{1}{l|}{50,00\%} & \multicolumn{1}{l|}{45,00\%} & \multicolumn{1}{l|}{30,00\%} & \multicolumn{1}{l|}{50,00\%} & 50,00\% \\ \hline
\multicolumn{1}{l|}{\textbf{Pedro Neto}} & \multicolumn{1}{l|}{30,00\%} & \multicolumn{1}{l|}{20,00\%} & \multicolumn{1}{l|}{40,00\%} & \multicolumn{1}{l|}{40,00\%} & \multicolumn{1}{l|}{20,00\%} & \multicolumn{1}{l|}{20,00\%} & 50,00\% \\
\textbf{} &  &  &  &  &  &  &  \\
\multicolumn{1}{l|}{\textbf{Pessoa / Semana}} & \multicolumn{1}{l|}{\textbf{14-May}} & \multicolumn{1}{l|}{\textbf{21-May}} & \multicolumn{1}{l|}{\textbf{28-May}} & \multicolumn{1}{l|}{\textbf{04-Jun}} & \multicolumn{1}{l|}{\textbf{11-Jun}} & \multicolumn{1}{l|}{\textbf{18-Jun}} & \textbf{25-Jun} \\ \hline
\multicolumn{1}{l|}{\textbf{André Nunes}} & \multicolumn{1}{l|}{35,00\%} & \multicolumn{1}{l|}{30,00\%} & \multicolumn{1}{l|}{40,00\%} & \multicolumn{1}{l|}{40,00\%} & \multicolumn{1}{l|}{40,00\%} & \multicolumn{1}{l|}{50,00\%} & 50,00\% \\ \hline
\multicolumn{1}{l|}{\textbf{Ana Catarina Sousa}} & \multicolumn{1}{l|}{20,00\%} & \multicolumn{1}{l|}{20,00\%} & \multicolumn{1}{l|}{40,00\%} & \multicolumn{1}{l|}{30,00\%} & \multicolumn{1}{l|}{30,00\%} & \multicolumn{1}{l|}{30,00\%} & 30,00\% \\ \hline
\multicolumn{1}{l|}{\textbf{Hugo Filipe Curado}} & \multicolumn{1}{l|}{20,00\%} & \multicolumn{1}{l|}{30,00\%} & \multicolumn{1}{l|}{40,00\%} & \multicolumn{1}{l|}{40,00\%} & \multicolumn{1}{l|}{40,00\%} & \multicolumn{1}{l|}{50,00\%} & 50,00\% \\ \hline
\multicolumn{1}{l|}{\textbf{Patricia Jesus}} & \multicolumn{1}{l|}{45,00\%} & \multicolumn{1}{l|}{30,00\%} & \multicolumn{1}{l|}{45,00\%} & \multicolumn{1}{l|}{30,00\%} & \multicolumn{1}{l|}{50,00\%} & \multicolumn{1}{l|}{30,00\%} & 70,00\% \\ \hline
\multicolumn{1}{l|}{\textbf{Pedro Neto}} & \multicolumn{1}{l|}{20,00\%} & \multicolumn{1}{l|}{20,00\%} & \multicolumn{1}{l|}{40,00\%} & \multicolumn{1}{l|}{30,00\%} & \multicolumn{1}{l|}{30,00\%} & \multicolumn{1}{l|}{30,00\%} & 30,00\% \\
\textbf{} &  &  &  &  &  &  &  \\
\multicolumn{1}{l|}{\textbf{Pessoa / Semana}} & \textbf{02-Jul} &  &  &  &  &  &  \\ \cline{1-2}
\multicolumn{1}{l|}{\textbf{André Nunes}} & 100,00\% &  &  &  &  &  &  \\ \cline{1-2}
\multicolumn{1}{l|}{\textbf{Ana Catarina Sousa}} & 80,00\% &  &  &  &  &  &  \\ \cline{1-2}
\multicolumn{1}{l|}{\textbf{Hugo Filipe Curado}} & 100,00\% &  &  &  &  &  &  \\ \cline{1-2}
\multicolumn{1}{l|}{\textbf{Patricia Jesus}} & 90,00\% &  &  &  &  &  &  \\ \cline{1-2}
\multicolumn{1}{l|}{\textbf{Pedro Neto}} & 80,00\% &  &  &  &  &  & 
\end{tabular}
\caption{Disponibilidade por pessoa por semana}
\label{dppps}
\end{table}

\begin{table}[]
\centering
\begin{adjustbox}{width=1.1\textwidth, center=\textwidth}
\label{my-label}
\begin{tabular}{l|c|c|c|c|c|c|c|c|c}
 & \multicolumn{1}{l|}{Modelos} & \multicolumn{1}{l|}{\begin{tabular}[c]{@{}l@{}}Testes\\ Modelos\end{tabular}} & \multicolumn{1}{l|}{Design} & \multicolumn{1}{l|}{\begin{tabular}[c]{@{}l@{}}Testes\\ Design\end{tabular}} & \multicolumn{1}{l|}{\begin{tabular}[c]{@{}l@{}}Construir e\\ Executar testes\end{tabular}} & \multicolumn{1}{l|}{MEAN} & \multicolumn{1}{l|}{LAMP} & \multicolumn{1}{l|}{Segurança} & \multicolumn{1}{l}{Redes} \\ \hline
Ana Sousa & E & E & E & E & E & B & M & M & M \\ \hline
André Nunes & M & M & M & M & E & E & B & M & E \\ \hline
Hugo Curado & M & M & M & M & E & M & E & E & E \\ \hline
Patricia Jesus & B & M & M & M & M & E & E & M & B \\ \hline
Pedro Neto & E & E & B & B & E & M & M & E & E
\end{tabular}
\end{adjustbox}
\caption{Tabela de competências da equipa}
\end{table}


\begin{table}[]
\centering
\begin{tabularx}{\textwidth}{X|c|X|X|c}
\textbf{Descrição} & \textbf{Disp.} & \textbf{Quando Necessário} & \textbf{Tempo Necessário} & \textbf{Categoria} \\ \hline
Bootstrap & \multicolumn{1}{c|}{100\%} & Durante o desenvolvimento do front end & Duração do projeto & off-the-shelf \\ \hline
CodeIgniter / Laravel & 100\% & Para o back end & Duração do projeto & off-the-shelf \\ \hline
Microsoft SQL Server & \multicolumn{1}{c|}{100\%} & Para o back end & Duração do projeto & off-the-shelf \\ \hline
MySql & 100\% & Para o front end & Duração do projeto & off-the-shelf \\ \hline
jQuery & 100\% & Para o front end & Duração do projeto & off-the-shelf \\ \hline
Angular & 100\% & Para o back end & Duração do projeto & off-the-shelf \\ \hline
Express & 100\% & Para o back end & Duração do projeto & off-the-shelf \\ \hline
Docker / Kubernets & \multicolumn{1}{c|}{100\%} & Durante o deployment do projeto &  & off-the-shelf \\
\end{tabularx}
\caption{Tabela de Recursos Software}
\label{TabRS}
\end{table}
\begin{table}[]
\centering
\begin{tabularx}{\textwidth}{l|c|X|c}
\textbf{Descrição} & \textbf{Disp.} & \textbf{Quando Necessário} & \textbf{Tempo Necessário} \\ \hline
Servidor Local & 500\% & Durante o desenvolvimento do projeto em ambiente local &  Duração do projeto\\ \hline
IDE de Desenvolvimento & 500\% & Durante todo o tempo de desenvolvimento &  Duração do projeto \\ \hline
Browser & 500\% & Durante todo o tempo de desenvolvimento &  Duração do projeto\\ \hline
Debugger do browser & 500\% & Durante todo o tempo de desenvolvimento & Duração do projeto  \\ \hline
Postman ou similar & 500\% & Durante todo o tempo de desenvolvimento & Duração do projeto \\
\end{tabularx}
\caption{Tabela de Recursos Ferramentas}
\label{TabF}
\end{table}

\pagebreak

\section{Processo de desenvolvimento de software}
A nossa escolha sobre o modelo de processo a utilizar recai sobre o modelo incremental porque é aquele que melhor se adapta à equipa e ao produto a desenvolver.\\
Visto ser um modelo constituido por mini cascatas conseguimos ter uma funcionalidade a 100\% mais rapidamente do que com qualquer outro modelo de processo. Isto será especialmente relevante nas reuniões com os clientes (professores), pois, assim conseguimos apresentar funcionalidades com maturidade de software e assim o retorno será mais preciso,ao contrário do que acontece com outros modelos de processo em que as funcionalidades nem sempre apresentam maturidade de software o que pode levar a um retorno mais vago.


\section{Organização da equipa}
O modelo de organização da equipa a usar é o modelo matricial. Esta escolha justifica-se principalmente pela modularidade do projecto em causa, o que elimina a organização horizontal, o facto de ser descentralizado é um factor a favor e elimina o modelo hierárquico que é centralizado.\\
Este modelo de organização da equipa funciona bem com o modelo de processo do projecto porque ambos se focam na modularidade.\\

\begin{table}[h]
\centering
\begin{adjustbox}{width=1.3\textwidth, center=\textwidth}
\label{my-label}
\begin{tabular}{l|l|l|l|l|l|l|l|l|l}
                                    & Especialista de Front-End & Architecture Designer & Especialista em Redes & Especialista em Segurança & Especialista de Back-End & Especialista em Testing & Analista & Gestor de Projeto & Cliente \\
Análise de Front-End                & C                         &                       &                       &                           &                          &                         & RA       & I                 &C         \\
Design de Front-End                 & R                       & RA                    &                       &                           &                          &                         &          & I                 &         \\
Implementação de Front-End          & RA                        &                       &                       &                           &                          &                         &          & I                 &         \\
Testing de Front-End                & R                     &                       &                       &                           &                          & RA                      &          & I                 &C I       \\
Análise de Back-End                 &                           &                       &                       &                           & C                        &                         & RA       & I                 & C        \\
Design de Back-End                  &                           & RA                    &                       &                           & R                      &                         &          & I                 &         \\
Implementação de Back-End           &                           &                       &                       &                           & RA                       &                         &          & I                 &         \\
Testing de Back-End                 &                           &                       &                       &                           & R                       & RA                      &          & I                 & CI       \\
Análise de Segurança                &                           &                       &                       & C                         &                          &                         & RA       & I                 & C        \\
Design de Segurança                 &                           & RA                    &                       & R                        &                          &                         &          & I                 &         \\
Implementação de Segurança          &                           &                       &                       & RA                        &                          &                         &          & I                 &         \\
Testing de Segurança                &                           &                       &                       & R                       &                          & RA                      &          & I                 & CI       \\
Análise de Simulador de Dados       &                           &                       &                       &                           &                          &                         & RA       & I                 & C        \\
Design de Simulador de Dados        &                           & RA                    &                       &                           &                          &                         &          & I                 &         \\
Implementação de Simulador de Dados &                           &                       &                       &                           &                          &                         &          & I                 &         \\
Testing de Simulador de Dados       &                           &                       &                       &                           &                          & RA                      &          & I                 &CI       \\
Análise de Redes                    &                           &                       & C                     &                           &                          &                         & RA       & I                 &   C      \\
Design de Redes                     &                           & RA                    & R                    &                           &                          &                         &          & I                 &         \\
Implementação de Redes              &                           &                       & RA                    &                           &                          &                         &          & I                 &         \\
Testing de Redes                    &                           &                       & R                    &                           &                          & RA                      &          & I                 & CI      
\end{tabular}
\end{adjustbox}
\caption{Tabela RACI}
\end{table}

\begin{table}[h]
\centering
\label{my-label}
\begin{tabular}{l|c|c|c|c|c}
                       & Ana Sousa & André Nunes & Hugo Curado & Patricia Jesus & Pedro Neto \\ \hline
Especialista Front-End & X         & X           & X           &                &            \\ \hline
Especialista Back-End  &           & X           & X           & X              &            \\ \hline
Architecture Designer  & X         &             &             &                & X          \\ \hline
Especialista Testing   & X         &             &             & X              & X          \\ \hline
Especialista Segurança &           & X           & X           &                & X          \\ \hline
Especialista Redes     & X         & X           &             &                & X          \\ \hline
Analista               & X         &             &             & X              &            \\ \hline
Gestor de Projecto     &           & X            &             &                &            \\
\end{tabular}
\caption{Tabela de competências}
\end{table}

\pagebreak

\section{Planeamento do Projeto}
Por favor consultar o ficheiro planeamentoPGP001.mpp.


\section{Gestão de Riscos}
\label{GR}
\begin{table}[h]
\centering
\begin{tabularx}{\textwidth}{c|X|c|c|c}
\# & Risco & Prob. & Impacto & Categoria \\ \hline
1                        & Equipa não familiarizada com tecnologias                            & 4                          & Projeto: 2, Produto: 4       & Projeto                        \\ \hline
2                        & Má estimação da complexidade                                        & 4                          & Projeto: 4, Produto: 3       & Projeto                        \\ \arrayrulecolor{red}\hline\arrayrulecolor{black}
3                        & Incompatibilidade entre as tecnologias                                 & 3                          & Projeto: 4, Produto: 4       & Projeto                        \\ \hline
4                        & Requisito mal entendido                                             & 3                          & Projeto: 4, Produto: 4       & Projeto                        \\ \hline
5                        & Software não apresenta maturidade (bugs)                            & 3                          & Projeto: 3, Produto: 4       & Técnico                        \\ \hline
6                        & Sobrecarga trabalhos (outras disciplinas)                           & 3                          & Projeto: 3, Produto: 3       & Projeto                        \\ \hline
7                        & Má configuração dos servidores & 2                          & Projeto: 5, Produto: 5       & Técnico                        \\ \hline
8                        & Prometer demasiado aos clientes(professores)                                 & 2                          & Projeto: 4, Produto: 4       & Negócio                        \\ \hline
9                        & Elemento da Equipa adoecer                                          & 2                          & Projeto: 3, Produto: 3       & Projeto   
\end{tabularx}
\caption{Tabela de Gestão de Risco}
\label{TabGR}
\end{table}
\begin{table}[h]
\centering
\begin{tabularx}{\textwidth}{c|X|X|X}
\# & Mitigação  & Monitorização & Gestão \\ \hline
1  & Dividir equipa em cada competência (Tabelas de Recursos); Documentação                                     & Acompanhar projetos e desenvolvimento de cada componente                                                     & Rodar pessoas, reunir e explicar caso haja alguém fluente \\ \hline
2  & Tendo em conta a estimação pessimista faz se uma maior divisão de componentes; Alargar prazo da componente & Acompanhar desenvolvimento de cada componente; Verificar atrasos; Acompanhar dificuldades no desenvolvimento & Colocar mais pessoas a trabalhar na componente         
\end{tabularx}
\caption{Tabela RMMM}
\label{TabRIMM}
\end{table}
\chapter{Conclusão}

Tendo como base a análise feita na cadeira de Conceção de Produto e tendo analisado um pouco mais a arquitetura estrutural das necessidades para a elaboração do projecto elaboramos este planeamento. A nossa solução foi pensada de modo a simplificar as interações dos utilizadores com os espaços da faculdade e ao mesmo tempo mantendo todo o controlo e segurança necessários. 
Com os primeiros valores do COCOMO podemos concluir que o projecto planeado é demasiado ambicioso para o tempo disponível. Perante o nosso modelo de desenvolvimento incremental, decidimos refazer os cálculos de modo a elaborarmos primeiamente os requisitos fundamentais para obter aprovação nas cadeiras de PTI e PTR. Assim sendo concluimos que o projecto é viável se ser executado no tempo pretendido. \\
Contudo, o grupo compromete-se a apresentar uma solução baseada nas funcionalidades descritas nos requisitos funcionais do projecto, dando prioridade às funcionalidades core. Com o trabalho realizado ao longo desta cadeira, reconhecemos um dos príncipais obstáculos à elaboração de um projecto, ou seja, um dos factores mais importantes é ter um bom planeamento que corresponda com as necessidades tecnicas, estruturais e tecnológicas do mesmo. Muitas das vezes a parte do planeamento pouco elaborada e por causa disso a execução do projecto pode ser a esperada.
\clearpage
\bibliography{BiliografiaPGP1718R1}{}
\bibliographystyle{IEEEtran}
\end{document}

